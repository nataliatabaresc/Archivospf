\documentclass{article}
\usepackage[utf8]{inputenc}

\title{Nacimiento de la computación moderna}
\date{}
\author{}

\begin{document}

\maketitle

El origen de las matemáticas es un Hazaña para toda la humanidad, sin embargo, en su estudio inicial se presentaron las matemáticas como algo intuitivo, lo que llevo a muchos matemáticos a cuestionarse y a estudiar estas más a fondo. El es logro que un matemático Georg Cantor, encontrara paradojas en estas matemáticas “intuitivas”, generando así la crisis de los fundamentos. Pero, Gracias a esta crisis la humanidad logro llegar muy lejos y descubrir los algoritmos y la computación.

En un inicio, la sociedad al no poder explicar ciertos conceptos matemáticos, se llegó a la conclusión, ahora descartada, existen nociones y preposiciones que provienen de la intuición y no de la lógica. Este pensamiento se llevó por muchos años, hasta llegar al sigo XIX. La crisis de los fundamentos, era un debate entre grandes matemáticos, en donde creaban la solución a las diferentes paradojas que se dieron a conocer de las matemáticas, además, esta crisis llevo a una terrible conclusión para la sociedad de ese entonces, diciendo que las matemáticas no son infalibles.

Gracias a las múltiples ideas de los matemáticos, la crisis dividió a estos en varias facciones, una de ellas eran los formalistas, en donde se encontraba David Hilbert; Planteando un método, llamada programa de Hilbert, consistía en demostrar que los sistemas axiomáticos poseen 3 propiedades. Primero, eran consistente, es decir, no se generaban contradicciones; Segundo, eran finitos, su demostración presentaba una serie de pasos que terminaban en algún momento; Tercero y último, eran completos, se refería a que el axioma era siempre o verdadero o falso.

En el siglo XX y después de muchos años de disputa, un matemático llamado Kurt Gödel, puso fin a la discusión, demostrando que ningún sistema puede ser consistente, finito y completo al mismo tiempo, mejor dicho, el programa de Hilbert era imposible de concluir. Gödel demostró de forma completa y meticulosa el teorema de incompletitud, acabando con el programa de Hilbert; Este teorema establece que, si un axioma no posee contradicciones, nunca se podrá demostrar su veracidad ni su falsedad.

Finalmente, Alan Turing se encargó de seguir con el legado de Gödel, continuando con sus investigaciones y con base en los teoremas de incompletitud que este descubrió. Alan es más conocido por participar en la primera guerra mundial, descifrando los mensajes codificados de los nazis con la famosa maquina enigma, primera computadora que existió. Años atrás había publicado un artículo que decía que los problemas no solo no tenían solución, sino que tampoco podríamos identificar cuáles eran. Por su teoría y manejo de enigma, a Alan Turing se le ocurrió la idea de demostrar esto por medio de “la maquina universal de Turing”. El mecanismo de esta máquina consistía en una cinta “infinita” dividida por casillas, un dispositivo que puede leer y sobrescribir en las casillas y también podía mover la cinta a la derecha y la izquierda; Con una sería de instrucciones dadas, la maquina podía sumar, restar, dividir, multiplicar y hacer cualquier tarea basada en una serie de pasos repetitivos, por más complejos que fueran.

Fue por esta problemática que se pudo llegar a la computación actual. La humanidad siempre encuentra la manera de dar un paso más allá, sin importar si comenzó de un error o un gran descubrimiento, por lo que podemos concluir que ninguna teoría por más absurda que sea no deberíamos considerarla error; Además, se tiene certeza de que La crisis llevo a una época de desarrollo enorme y avances continuos por lo que vivimos hoy en día.

\end{document}
